\documentclass[12pt]{article}
%%%%%%%%%%%%%%%%%
% Imported Packages
%%%%%%%%%%%%%%%%%
\usepackage{setspace}
\usepackage{url}
\usepackage{multicol}
\usepackage{etoolbox}
\usepackage{relsize}
\patchcmd{\thebibliography}
  {\list}
  {\begin{multicols}{2}\smaller\list}
  {}
  {}
\appto{\endthebibliography}{\end{multicols}}

\usepackage{titlesec}
\usepackage{flushend}
\expandafter\def\expandafter\UrlBreaks\expandafter{\UrlBreaks%  save the current one
  \do\\\do-}

\usepackage{soul}

\usepackage{graphicx}
\graphicspath{ {images/} }

\usepackage{capt-of}

%%% PAGE DIMENSIONS
\usepackage[top=4.5cm, bottom=4.5cm, left=2cm, right=2cm]{geometry}
\usepackage{geometry} % to change the page dimensions
\geometry{letterpaper}

\begin{document}


%%%%%%%%%%%%%%%
% Title
%%%%%%%%%%%%%%%
\title{\vfill LiteCrypto} %\vfill gives us the black space at the top of the page
\author{
Team AES - Andrew Wang, Sam (Jiewen) Wu, and Elton Yang \vspace{10pt} \\
CPE 458: Current Topics in Computer Systems (Cryptographic Engineering)  \vspace{10pt} \\
Dr. Zachary Peterson \vspace{10pt} \\
}
\date{December 12, 2014} %Or use \today for today's Date

\maketitle

\vfill  %in combination with \newpage this forces the abstract to the bottom of the page
\begin{abstract}
\end{abstract}
\thispagestyle{empty} %remove page number from title page, but still keep it as pg #1
\newpage
\thispagestyle{empty}
\tableofcontents
\thispagestyle{empty}
\mbox{}
\newpage

%%%%%%%%%%%%%%%%%%%%
%%% Known Facts  %%%
%%%%%%%%%%%%%%%%%%%%
\begin{multicols}{2}
\section{Introduction}
\subsection{CubeSat}
CubeSat is a international collaborative project of over 40 educational systems and private firms. A CubeSat is a satellite that is 10 cm3 and with a mass of 1.33 kg. 
\subsection{PolySat}
PolySat is Cal Poly’s branch of CubeSat, and was established since 1999. Dr. Bellardo is the current advisor for PolySat. They have eight launched missions and three are currently in development. 
\section{Motivation and Background}
Our group decided on creating a CubeSat Crypto Library for our Crypto Engineering Final Project. 
\subsection{Background}
Dr. Bellardo has discussed with our team that the future of CubeSat is to allow them to move in space. However, their current communications from home base to the satellite is encrypted and unauthorized.  Therefore, an adversary who wishes to harm the CubeSat, or harm another object using the CubeSat, may sniff a command packet to the satellite and control it. Dr. Bellardo has given us a set of requirements that the library should have.
\subsection{Requirements}
\subsubsection{Binary Size}
Due to the small size of the CubeSat and the large amount of things it does, there is not a lot of storage space for the library. Therefore, the binary size needed to be around 50 kilobytes. Because the library would be part of its main functions, it will also need to be small for a quick boot time for the operating system. 
\subsubsection{Data Transfer}
Our library can not use transfer large amount of data between home base and the satellite. AX.25 speed limits on data are rarely higher than 9,600 bits/s and are usually 1,200 bits/s. The satellite also does not return an acknowledgement packet after successful retrieving a packet. Therefore, we are restricted on which scheme to use for our crypto library. Another issue is the connection time with the CubeSat is at most 15 minutes an hour due to the satellite orbiting the Earth. Therefore, we cannot use schemes that require a constant connection with the satellite. 
\subsubsection{Encryption and Authentication}
Our library should allow any user to have the option to encrypt or authenticate their packets. It should have 3 functions: No security, one way authentication, and authenticated encryption. Using the diagram below is how Dr. Bellardo described how we should encrypt and authenticate. \\
\begingroup
    \centering
    \includegraphics[width=7cm]{researchDiagram.png}
\endgroup
We were told to only encrypt the data portion of the packet, and only authenticate the IPv4 and UDP headers along with the data.
\section{Related Work/Research}
\subsection{CyaSSL}
\subsection{NaCl}
\subsubsection{TweetNaCl}
\section{Our Work}
\section{Evaluation and Analysis}
\section{Conclusion}
\end{multicols}
\end{document}
